\documentclass{beamer}

\usepackage[english]{babel}
\usepackage[utf8]{inputenc}
\usepackage{amsfonts}
\usepackage{amssymb}
\usetheme{Copenhagen}
\usepackage{amsthm}
\usepackage{amsmath}
\usepackage{amsopn} 
\usepackage{amsmath,amssymb,lmodern}
\usepackage{pgf}  
\usepackage{textpos}

\addtobeamertemplate{navigation symbols}{}{%
    \usebeamerfont{footline}%
    \usebeamercolor[fg]{footline}%
    \hspace{1em}%
    \insertframenumber/\inserttotalframenumber  
}
%\logo{\pgfputat{\pgfxy(-1,7)}{\pgfbox[center,base]{\includegraphics[width=1.0cm]{logo.png}}}}

\addtobeamertemplate{frametitle}{}{%
\begin{textblock*}{100mm}(\textwidth,-1cm)
\includegraphics[height=1cm,width=1cm,keepaspectratio]{logo.png}
\end{textblock*}}

\title{Monitoring Helium storage with a LV-MaxSOnar-EZ1 sonar sensor}
\subtitle{From Sensor to Report - 1FA349}
\author[ Xabier García Andrade]{\includegraphics[height=3cm,width=3cm]{logo.png}\\  Xabier García Andrade}



\begin{document}
	\frame {
		\titlepage
	}
	\frame {
		\frametitle{FREIA Laboratoriet}
		\begin{columns}[T]
		\begin{column}[T]{5cm}
		\begin{block}{Purpose}
      Accelerator development and light generation by charged particles. 
   \end{block}
   		\begin{alertblock}{How are particles accelerated?}
		RF cavities and magnets to deflect their trajectories.
		\end{alertblock}
		\includegraphics[height = 3cm]{IMG_20181108_164216.jpg}
		\end{column}
		\begin{column}[T]{5cm}
		\begin{exampleblock}{Cooling down the equipment.}
		\includegraphics[height = 3cm]{IMG_20181108_164125.jpg}
		\end{exampleblock}
		\begin{itemize}
		\item Helium Liquifier.
		\item 2 K cryostats.
		\end{itemize}
		\end{column}
		\end{columns}
	}
	\frame {
		\frametitle{Motivation of the Project}
		\begin{columns}[T]
		\begin{column}[T]{5cm}
		\begin{block}{Scope}
      Monitoring the amount of helium contained in the gas balloon.
   \end{block}
%   		\begin{figure}
%    \subfloat{{\includegraphics[height = 3.5cm]{scheme.png} }}
%    \qquad
%    \subfloat{{\includegraphics[height = 3.5cm]{IMG_20181108_164636.jpg} }}
%    \label{fig:example}
%\end{figure}
	\begin{figure}
   \includegraphics[width=0.8\textwidth]{scheme.png}
   \hfill
   \includegraphics[width=0.475\textwidth]{IMG_20181108_164636.jpg}
\end{figure}
		\end{column}
		\begin{column}[T]{5cm}
		\begin{exampleblock}{How do we achieve this?}
		Using ultrasounds to measure the distance to the surface from different angles.
		\end{exampleblock}
		\includegraphics[width = 3.5cm]{bat.jpg}
		\includegraphics[height = 3cm]{batman.jpeg}
		\end{column}
		\end{columns}
	}
	\frame{
		\frametitle{Overview}
		\includegraphics[width=1.0 \textwidth]{diagram.png}
		}
	\frame{
		\frametitle{Hardware}
		\framesubtitle{Sensors and actuators}
		\begin{columns}[T] 
     \begin{column}[T]{5cm}
  	 \begin{block}{Sonar}
      Emits a short ultrasonic sound burst and records the duration until the echo arrives.
  	 \end{block}
  	 \begin{alertblock}{Several Output Modes}
  	 We use Pulse Width.
  	 \end{alertblock}
  	 \begin{exampleblock}{Specifications.}
  	 \begin{itemize}
  	 \item Resolution of 1 in.
  	 \item Maximun Range of 254 in.
  	 \item Operates from 2.5 V to 5.5 V and draws 2 mA.
  	 \end{itemize}
  	 \end{exampleblock}
     \end{column}
     \begin{column}[T]{6cm} 
          \includegraphics[height=3cm]{sonar_with_diagram.png}
			\vspace{1.0cm}
          \begin{figure}[!b]
          \includegraphics[width= 5cm]{pulse_width.png}
          \end{figure}
     \end{column}
     \end{columns}
		}
		\frame{
		\frametitle{Hardware}
		\framesubtitle{Sensors and actuators}
		\begin{columns}[T] 
     \begin{column}[T]{5cm}
     \begin{block}{Servo}
      Motor with a position encoder.
  	 \end{block}
  	 \begin{exampleblock}{How to control it.}
  	 By sending a PWM signal, the width of the pulse determines the position of the servo.
  	 \end{exampleblock}
  	 \begin{figure}[b]
  	 \includegraphics[width = 5.5cm]{servo_pw.png}
  	 \end{figure}
     \end{column}
     \begin{column}[T]{5cm} 
          \includegraphics[height=3cm]{servo_with_diagram.png}
          \begin{alertblock}{Mechanical Attachment.}
          \includegraphics[height=3cm]{attachment.jpg}
          \end{alertblock}
     \end{column}
     \end{columns}
		}
	\frame{
		\frametitle{Hardware}
		\framesubtitle{Arduino Uno}
		\begin{columns}[T]
	\begin{column}[T]{5 cm}
	\begin{itemize}
	\item Microcontroller board 
	\item 14 digital input/output pins 
	\item USB connection 
	\item 16 MHz quartz crystal \\~\\
	\end{itemize} 
	\textbf{Reads the raw signal from the sensor and transforms it into a relevant physical quantity.}
	\end{column}	
		\begin{column}[T]{5cm} 
		  \begin{figure}[t]
		  \includegraphics[width= 3 cm]{diagram_arduino.png}
		  \end{figure}
          \includegraphics[height=5cm]{arduino_uno.jpg}
     \end{column}
     \end{columns}
		
	}
		\frame{
		\frametitle{Interface}
		\begin{columns}[T]
		\begin{column}[T]{6.5 cm}
		\begin{itemize}
		\item Ethernet shield allows the arduino to communicate via Ethernet.
		\item More reliable and faster than Bluetooth.
		\end{itemize}
		\includegraphics[width = 7cm]{ethernet_snippet.png}
		\end{column}
		
		\begin{column}[T]{5 cm}
				  \begin{figure}[t]
		  \includegraphics[width= 3 cm]{diagram_ethernet.png}
		  \end{figure}
		\includegraphics[height=4cm]{eth_shield.jpg}
		\end{column}
		\end{columns}
		
	}
		\frame{
		\frametitle{Epics Control System}
		\begin{columns}[T]
		\begin{column}[T]{6 cm}
		\begin{block}{What is it?}
		Software tools used to create distributed soft real-time control systems for scientific instruments such as a particle accelerators.
		\end{block}
		\begin{exampleblock}{Specifications}
		\begin{itemize}
		\item Communication: Client/Server and Publish/Subscribe techniques.
		\item Servers: IOCs (Input/Output Controllers).
		
		\end{itemize}
		\end{exampleblock}
		\end{column}
		\begin{column}[T]{5 cm}
		\includegraphics[width=1.0\textwidth]{kstar_epics.jpg}
		\includegraphics[width=1.0\textwidth]{slac_epics.jpg}
		\begin{exampleblock}{}
		\begin{itemize}
		\item Publish: Using Channel Access (CA) network protocol.
		\end{itemize}
		\end{exampleblock}
		\end{column}
		\end{columns}
	}
	\frame{
		\frametitle{Software}
		\framesubtitle{Arduino Protocols}
		\begin{columns}[T]
		\begin{column}[T]{5cm}
		\begin{block}{Setup:}
		At the beginning we need to specify the pins that we will be using, as  well as initialize the Ethernet server
		\end{block}
  		 \end{column}
  		 \begin{column}[T]{7cm}
  			 \includegraphics[height=4cm]{setup.png}

  		 \end{column}
		
		\end{columns}
	}
	\frame{
		\frametitle{Software}
		\framesubtitle{Arduino Protocols}
		\begin{columns}[T]
		\begin{column}[T]{5cm}
		\begin{block}{Function where the distance is measured:}
		\end{block}
		\includegraphics[height=4cm]{code_measure.png}
  		 \end{column}
  		 \begin{column}[T]{5cm}
  		 	 		\textbf{Three different commands:}
		 \begin{exampleblock}{"Scan"}
   			Servo rotates while measuring distance in every position
   		 \end{exampleblock}
  			 \includegraphics[height=3.2cm]{scan_snippet.png}

  		 \end{column}
		
		\end{columns}
	}
	\frame{
		\frametitle{Software}
		\framesubtitle{Arduino Protocols - II}
		\begin{columns}[T]
		\begin{column}[T]{5.5cm}
		 \begin{exampleblock}{"Go to"}
  		 Servo moves to the selected position
   			\end{exampleblock}
   		  \includegraphics[height=2cm]{go_to_snippet.png}
  		 \end{column}
  		 \begin{column}[T]{6cm}
 		    \begin{exampleblock}{"Measure"}
   		 Measures distance without moving
  		 \end{exampleblock}
  			 \includegraphics[height=2.5cm]{measure_snippet.png}
  		 \end{column}
		\end{columns}
	}
	\frame{
		\frametitle{Software}
		\framesubtitle{Arduino Protocols - III}
		\begin{columns}[T]
		\begin{column}[T]{5 cm}
		\begin{block}{Waveforms}
		After scaning, all the values will be stored in a matrix. Since we want to assign these values to a process variable in epics, we print them as a waveform.
		\end{block}
		\end{column}
		\begin{column}[T]{6 cm}
		\includegraphics[height=5cm]{waveforms.png}
		\end{column}
		\end{columns}
	}

	\frame{
		\frametitle{Software}
		\framesubtitle{Epics Control System}
		
		\begin{columns}		
		 \begin{column}[T]{6cm}
		 	\begin{alertblock}{Database File}
		 	Establishes how every variable is stored.
		 	\end{alertblock}
  			 \includegraphics[width = 6.5cm]{epics_db_snippet.png}
 			 \begin{figure}[b]
 			 \includegraphics[width = 2.5cm]{diagram_epics.png}
 			 \end{figure}
  		 \end{column}
  		 \begin{column}[T]{4cm}
  		 \begin{alertblock}{Protocol File}
  		 Defines functions that will be used to obtain the variables:
  		 \end{alertblock}
  		 \includegraphics[width = 4.5cm]{epics_proto_snippet.png}
  		\end{column}
  		\end{columns}
		
	}
	\frame{
		\frametitle{Prototype}
		\begin{columns}[T]
		\begin{column}[T]{5 cm}
		\includegraphics[width = 4.5cm]{60graosteito.png}
		\includegraphics[width = 4.5cm]{plot_snipp.png}
		\end{column}
		\begin{column}[T]{5 cm}
		\includegraphics[width = 4.5cm]{scan_python.png}
		\end{column}
		\end{columns}
	}
		
	\frame{
	    \frametitle{Future Enhancements}
	    \begin{block}{In the next future...}
	    \begin{itemize}
	    \item Array of Sonars. 
	    \item Sonar with better resolution. 
	    \item Kinect for 3D reconstruction.
	    \item Guide Rail. 
	    \end{itemize}
	    \begin{figure}[H]
	    \includegraphics[width = 6 cm]{cern_random.jpg}
	    \end{figure}
	    \end{block}


	}
\end{document}